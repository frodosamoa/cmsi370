\documentclass[11pt, oneside]{article}   
\usepackage{geometry}                	
\geometry{letterpaper}                   	
\usepackage{graphicx}						
\usepackage{amssymb}

\title{Heuristic Evaluation}
\author{Andrew Kowalczyk}

\begin{document}
\maketitle

In this heuristic evaluation, I mention the main three different types of customers that enter any given shopping site which are as follows:
\begin{enumerate}
\item Customers who know what they want to buy 
\item Customers who do not know what they want to buy
\item Customers who have considered buying something, yet have not yet purchased said item
\end{enumerate}
I will refer to these types of customers throughout the evaluation.

In terms of the five usability metrics, I define \textit{learnability} as the time to learn, \textit{efficiency} as speed of performance, \textit{errors} as rate of errors, \textit{memorability} as retention over time, and \textit{satisfaction} as subjective satisfaction.
\pagebreak

\section{Amazon}

\subsection{Mental model}

Amazon's mental mondel is quite clear from the moment you land on their homepage. In order to illustrate the diesigner's mental models, I will focus on just the homepage.
Amazon's designers carried this mental model about the types of customers that they are serving when they designed their website. Each of the types of customers can also be logged in or not, and Amazon accounted for these types in their design.

The \textbf{search bar} is there if the customer has a specific product in mind to buy. This allows the customer for a very refined search of what they are purchasing. This works extremely well because they designer has the mental model that they customer can simply type exactly what they are searching for, and this allows the customer to do so. There even is a dropdown on the left of the input field to search within a specific category. The search terms also tries to autocomplete the search so the customer can type less. A customer does not need to be logged in for this to work as Amazon has designed it.

The \textit{Shop by \textbf{Department}} dropdown is there for the customers who want to shop but do not know what they want to buy. Their main cateogies are listed here with the most popular subcategories for those given categories. Yet, at the bottom of this dropdown, one can see a link to the \textit{Full Store Directory}. Thanks to both of these options, a customer who does not know what they want to buy has no problem figuring this out. Similarly to the search bar, a customer does not need to be logged in for this to be functional.

The \textbf{horizontal lists} have slightly different functionality depending on if a customer is logged in or not. In either case, many horizontally lists show up below the dropdown on the left which are mainly geared towards a customer who does not have a clear idea of what they exactly want to buy. If the customer is not logged in, the lists can be \textit{Included with Prime Membership at No Additional Cost}, \textit{What Other Customers Are Looking At Right Now}, etc. If a customer is logged in the lists change to \textit{More Items to Consider}, \textit{Related to Items You've Viewed}, \textit{New For You}, and \textit{Inspired by Your Shopping Trends}. They usually also include a list that begins with the words \textit{Reccomendations for You in} which is followed by a category of products that you might have been looking at recently. The horizontal lists can help a customer make a choice for any occasion, regardless if they are looged in or not.

The \textbf{search bar}, \textit{Shop by \textbf{Department}} dropdown, and \textbf{horizontal lists} account for all the types of customers that enter Amazon's site. The mental models of the designers work incrdibly well because they have figured out a way to approach every customer.

\subsection{Five key usability metrics}

\subsubsection{Learnability}
In terms of the shopping process, Amazon's site is designed well because it takes little to no time at all to learn how to navigate it. The layout remains consistent throughout the every step of the way. As I mentioned previously, their website appeals to all types of customers which makes it very learnable. Through out every step of the way, the action required to get to the next step of the process is always on the right of the web page. The shopping process is very learnable.

Yet, Amazon's account page takes some more time to learn. They inlcude many links of just text and options that most common shoppers do not use like \textit{Manage Bulk Gift Card Orders} and \textit{Manage Textbook Rentals}. It takes time to simply get familar with the layout.

All in all, their interface is fairly learnable and takes not too much time to learn.

\subsubsection{Efficiency}
Amazon's checkout process is quite error-less. In order to buy a product, Amazon offers a few options for completeing a purchase for its customers. 

First, they allow the customer to add their product to their shopping cart. This process is efficient because, as mentioned previously, each subsequent step is always located in the same area.

Second, they also allow customers to buy with 1-Click\textregistered\hspace{2 mm}(I didn't want them to get mad). This is quite incredibly efficient. It holds true to its name and really allows for a single click to purchase a product. Before clicking buy, it allows for the user to choose which address they ship to from that are of the screen.

Overall, the whole process of using their website is efficient.

\subsubsection{Errors}
A potential error that may arise during shopping using the 1-Click\textregistered\hspace{2 mm}method, is that each location saved in a user's account can have a default payment method. If a user forgets which payment method they have chosen for a given location to ship to, this can charge a different payment method than they might have expected. Besides this, their website is error-free.

\subsubsection{Memorability}
In terms of memorability, Amazon's site utilizes the idea of keeping the link or button that completes an action on the right side of the page. This make it memorable because any user can expect to see similar actions on that side fo the page. Overall, Amazon's site is memorable.

\subsubsection{Satisfaction}
Amazon approaches satisfaction in an incredibly reasonable fashion. Since their interface is learnable, efficient (for the most part), error free, and memorable. It is no suprise that Amazon's satisfaction is high.

\pagebreak

\section{eBay}

\subsection{Mental model}

Although specializing in a different kind of market than Amazon, eBay's mental models have many similarities to Amazon.

Search bar autocompletes.

\subsection{Five key usability metrics}
\subsubsection{Learnability}
\subsubsection{Efficiency}
\subsubsection{Errors}
\subsubsection{Memorability}
\subsubsection{Satisfaction}

\subsection{Interaction design guidelines, principles, \& theories}

\pagebreak

\section{Shopzilla}

\subsection{Mental model}

Looking at Shopzilla's main site, you can see that in their mental model users are inclined to type out what they are searching for. They are presented with a table of categories at the bottom of the page.

Shopzilla's search does not auto complete like Amazon and eBay's search bars do.

\subsection{Five key usability metrics}
\subsubsection{Learnability}
\subsubsection{Efficiency}
\subsubsection{Errors}
\subsubsection{Memorability}
\subsubsection{Satisfaction}

\subsection{Interaction design guidelines, principles, \& theories}


\end{document}